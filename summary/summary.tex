\documentclass{paper}


\usepackage[utf8]{inputenc}



\usepackage[english]{babel}
\usepackage[a4paper, margin=0.8in]{geometry}

\usepackage{parskip}

\usepackage{amsthm}
\usepackage{amssymb}
\usepackage{amsmath}




% variables you can adjust
\newcommand{\titleVar}{\large Seminar Report on\\ \LARGE TCUDB: Accelerating Database with Tensor Processors \\{\large {by Yu-Ching Hu, Yuliang Li, Hung-Wei Tseng}}
}

\newcommand{\authorVar}{David Raphael Zollikofer}
% end variables you can adjust


% last configuration before document starts
\title{\titleVar}
\author{David Raphael Zollikofer\\ ETH Zürich}
\date{\today}



\usepackage{mathpazo}

% optional math environments



\begin{document}
	
\twocolumn[\maketitle 
\hrule 

\begin{abstract} advances in hardware such as cuda have enabled plenty of new applications (cite deep learning). This report will show how these new accelerators can be used .. based on the paper ... . This approach will be compared to other literature in the area. A thorough .. of the proposed methods performance will be given as well as highlight the methods problems. \end{abstract}

\hrule\bigskip
]

	
	\section{Introduction}
	Goal of using tensors, how tensors work 
	
	\section{Using Tensor Cores to Accelerate Databases}
	\subsection{The Tensor Abstraction}
	
	\subsection{Mapping Database Operators to Tensor Operations}
	
	\subsection{End-to-End Pipeline}
	
	\section{Comparison to Related Work}
	
	\cite{he2022query}lol \cite{hu2021tcudb}
	  \cite{sun_2022}
	if we use tensors we must use 
	
	\section{Performance Evaluation}
	
	clearly good on ...
	
	very expected
	
	\section{Shortcomings}
	
	monetDB is super old, why not compared to Vectorwise, questionable 
	
	TPC-H benchmark like others
	
	clearly inefficient? 
	
	how many runs, average?
	
	
	\section{Conclusion}
	
\bibliographystyle{plain}
\bibliography{bibliography}
	

\end{document}
